\documentclass[12pt,oneside,a4paper]{book}

%%%%%%%%%%%%%%%%%%%%%%%%% preamble %%%%%%%%%%%%%%%%%%%%%%%%%
%% 引用宏包 %%
\usepackage{ctex}
\usepackage{titlesec}
\usepackage{amsmath,amssymb,amsthm,multicol,siunitx,graphicx,enumitem,booktabs,appendix}
\usepackage{booktabs}
\usepackage{tabularx,subcaption,arydshln}
\usepackage{tikz}
\usepackage{arydshln}
\usetikzlibrary{calc}
\usetikzlibrary{positioning}
\usetikzlibrary{matrix, positioning, arrows}
\usepackage[edges]{forest}
\usepackage{pgfplots}
\pgfplotsset{compat=1.8}
\usepackage{imakeidx}
\usepackage{framed}
\usepackage{makecell}
\usepackage{multirow}
\usepackage[export]{adjustbox}

%% 超链接与引用 %%
\usepackage{hyperref}
\hypersetup{
    hidelinks, % 取消红框
    colorlinks=true,
    linkcolor=blue,
    citecolor=pink
}
\renewcommand{\figureautorefname}{图}
\renewcommand{\tableautorefname}{表}
\renewcommand{\equationautorefname}{公式}
\def\chapterautorefname~#1\null{第~#1~章\null}
\def\sectionautorefname~#1\null{第~#1~节\null}
\def\subsectionautorefname~#1\null{第~#1~小节\null}


%%%%%%%%%%%%%%%%%%%%%%%%%%% info %%%%%%%%%%%%%%%%%%%%%%%%%%%

%% 文档信息及标题 %%
\title{凝聚态物理}
\author{}
\date{\today}


%%%%%%%%%%%%%%%%%%%%%%%%% document %%%%%%%%%%%%%%%%%%%%%%%%%
\begin{document}
\maketitle          %标题

\frontmatter
\chapter{前言}
\textbf{关于本书}

作为笔者对\textbf{凝聚态物理}的学习笔记,本书涵盖《固体物理》、《半导体物理》、《固态器件》三门学科。

在第一部分《固体物理》中,我们将介绍\textbf{固态晶体}的晶格结构、倒格子、晶格振动、能带理论等内容;

在第二部分《半导体物理》中,我们将介绍\textbf{半导体}的电子状态、掺杂与缺陷、载流子的统计分布、导电性、非平衡载流子等内容;

在第三部分《固态器件》中,我们将介绍PN结、肖特基二极管、双极性晶体管、异质结晶体管、MOS场效应管等\textbf{半导体器件}。

\hspace{2em}

\textbf{内容参考}

整理笔记过程中,参考了普渡大学\href{https://nanohub.org/courses/ECE606/2020x/outline}{ECE 606:固态器件}课程。

笔记亦可作为电子科学与技术专业学生的参考资料。

\hspace{2em}

\textbf{致谢}

由于笔者能力有限,内容势必存在不足与缺漏,敬请斧正。

\tableofcontents    %目录

\mainmatter

\part{固体物理}
\chapter{晶体结构与性质}
\section{常见晶体结构}
\section{晶体的平移对称性}
\section{晶体的方向性}
\section{晶体的宏观对称性}

\chapter{倒格子与波矢空间}
\section{倒格子}
\section{布里渊区}
\section{X射线衍射}

\chapter{晶格振动}
\section{一维单原子链的振动}
\section{一维双原子链的振动}
\section{三维晶格的振动}

\chapter{能带理论}
\section{自由电子近似}
\section{近自由电子近似}
\section{紧束缚近似}

\part{半导体物理}
\chapter{半导体的电子状态}
\chapter{半导体的掺杂与缺陷}
\chapter{半导体的载流子统计分布}
\chapter{半导体的导电性}
\chapter{半导体的非平衡载流子}

\part{半导体器件}
\chapter{PN结}
\chapter{MOS场效应管}

\backmatter

\end{document}